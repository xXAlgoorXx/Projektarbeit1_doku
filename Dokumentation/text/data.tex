\chapter{Hexagon Dataset
    \label{chapter:dataset}}
Most of the information in this chapter is taken from Lia Winkler's master's thesis.  
The dataset was collected using the terrestrial laser scanner RTC 360 from Leica at 32 different locations.  
In total, this dataset consists of 821 clean images, with the number of images per location ranging from 4 to 423.  
All images are 360-degree panoramas with a resolution of 8192 \(\times\) 4096 pixels.  
The images are divided into the following 5 classes:

\begin{itemize}
    \item Indoor architectural
    \item Indoor construction
    \item Outdoor construction
    \item Outdoor Urban
    \item Forest
\end{itemize}

\noindent
To increase the accuracy of the algorithm, the classes were rephrased and divided into subclasses.  
These subclasses can be seen in \cref{tab:dataset:subclasses}.

\begin{table}[!ht]
    \centering
    \begin{tabular}{ccc}
    \toprule
    \textbf{1. Original}& \textbf{2. Rephrased}& \textbf{3. Subclasses}\\ \midrule
    Construction (in) & construction site & \makecell{construction site,\\ mining, tunnel}\\ \hline
    Architectural (in)& architectural& \makecell{architectural, office,\\ residential, school,\\ manufacturing, cellar,\\ laboratory} \\ \hline
    Construction (out)& construction site & construction site \\ \hline
    Urban (out)& town & \makecell{town, city,\\countryside, alley,\\ parking lot} \\ \hline
    Forest (out)& forest& forest \\
    \bottomrule
    \end{tabular}
    \caption{Rephrased terms and their subclasses
        \label{tab:dataset:subclasses}}
\end{table}

\begin{table}
    \centering
    \begin{tabular}{cc}
    \toprule
    \textbf{Class}& \textbf{Picture Classes}\\ \midrule
    Construction (in) & 9,13,39,12 \\ \hline
    Architectural (in)& 7,10,18,27,29,32,36,1,28,6,33,40,30,31,24\\ \hline
    Construction (out)& 8,16,22\\ \hline
    Urban (out)& 2,20,38,26,15,42,44,4,23\\ \hline
    Forest (out)& 17\\
    \bottomrule
    \end{tabular}
    \caption{Picture Classes of the hexagon dataset
        \label{tab:dataset:piccoding}}
\end{table}

The labels of the pictures are encoded in the name of the picture. 
The first number indicates the location where the picture was taken, 
while the second number is a sequential number for that location. 
\cref{tab:dataset:piccoding} shows the correspondence between locations and classes. 
For example, the picture in \cref{fig:dataset:examplepic} belongs to the "Construction (in)" class.

It is important to mention that the dataset is highly imbalanced. 
Nearly 80\% of the panorama images belong to the "Indoor Architectural" class. 
This means that a "naive" model can achieve an accuracy of 80\% simply by labeling every input as "Indoor Architectural." 
It is crucial to consider this fact when evaluating the performance of a model.

\begin{figure}
    \centering
    \includegraphics[width=\textwidth]{Images/Dataset/panorama_00009_0015.jpg}
    \caption{Example picture from the hexagon dataset [panorama\_00009\_0015.jpg]}
    \label{fig:dataset:examplepic}
\end{figure}

