%
% Abstrakt.tex
%
% (c) 2023 Jakob Gierer & Lukas Schöpf, OST Ostschweizer Fachhochschule
%

\chapter*{Abstrakt}
	Drohnen sind aufgrund ihrer einfachen Steuerbarkeit und Verfügbarkeit für jedermann nutzbar.
	Die zunehmende Beliebtheit von Drohnen in den letzten Jahren hat zu Sicherheitsbedenken hinsichtlich Privatsphäre, Spionage und militärischem Missbrauch geführt.
	Flüge in der Nähe von Flughäfen und Landeplätzen können auch zu Störungen und Verzögerungen im Flugverkehr führen.
	Daraus ergibt sich die Notwendigkeit eines Systems, das Drohnen zuverlässig orten und erkennen kann.
	Bisherige Systeme arbeiten mit Radar-, Funk- und Kamerasensoren, die alle ihre Grenzen haben.
	Menschen sind von Natur aus in der Lage, Objekte zu orten und zu identifizieren.
	Ziel dieser Arbeit ist es ein Algorithmus zu entwerfen, welcher Drohnen von anderen Umgebungsgeräuschen unterscheidet.
	\vspace{10pt}
	
	Der Ansatz des Projekts besteht darin, mit Hilfe von künstlichen neuronalen Netzen Muster in den aufgenommenen Audiodaten zu erkennen.
	Die Audiospuren werden mit einem dreidimensionalen Mikrofonarray mit 32 Mikrofonen aufgenommen und mit einem Beamforming-Algorithmus, der aus den 32 Audiospuren eine gerichtete Audiospur erzeugt, verarbeitet.
	Das Hauptziel dieser Arbeit ist die Entwicklung eines Algorithmus zur Erkennung von Drohnen-Geräuschen in Audiodaten.
	
	Mit einer eigenen Augmentierungssoftware wird der Datensatz für das Training eines Netzes vergrössert.
	Dieser Datensatz wird mit einem selbst entworfenen CNN trainiert, welches als Input die MFCC nutzt.
	Eine Echtzeitanwendung analysiert die Audiospuren mit dem zuvor trainierten Modell und berechnet, ob es sich bei der Audiospur um ein Drohnen-Geräusch handelt.
	Die erkannten Drohnen werden auf einer Benutzeroberfläche visualisiert.
	\vspace{10pt}
	
	Die durchgeführte Klassifizierung von Drohnen erreichte einen Precision von 92,1\% und Recall von 96,9\% mit den bereitgestellten Outputdaten aus dem Beamforming.
	Diese Ergebnisse entsprechen den an das Projekt gestellten Erwartungen.
	Im Rahmen dieses Projekts werden die Möglichkeiten einer akustischen Detektion von Drohnen demonstriert.
	
	