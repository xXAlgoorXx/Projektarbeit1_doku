%
% main.tex
%
% (c) 2024 Lukas Schöpf, OST Ostschweizer Fachhochschule
%


\documentclass[hidelinks]{report}
% öffnet packages.tex
%
% packages.tex -- TeXfiles
%
% (c) 2023 Jakob Gierer & Lukas Schöpf, OST Ostschweizer Fachhochschule
%


%packages
\usepackage[english]{babel}
\usepackage[utf8]{inputenc}
\usepackage{csquotes}
\usepackage[T1]{fontenc}
\usepackage[backend=biber, style=ieee]{biblatex}
\usepackage{hyperref}
\usepackage{amsmath,amscd}
\usepackage{amssymb}
\usepackage{amsfonts}
\usepackage{amsthm}
\usepackage{amsmath}
\usepackage{makecell}
\usepackage{txfonts}
\usepackage{graphicx}
\usepackage{tabularx}
%\usepackage{subcaption}
\usepackage{subfig}
\usepackage[many]{tcolorbox}
\usepackage[acronym]{glossaries}
\usepackage{fancyhdr}
\usepackage[toc,page]{appendix}
\usepackage[headheight=12.1pt]{geometry}
\usepackage[english]{cleveref}
\usepackage{lmodern}
\usepackage{booktabs}
\pagestyle{fancy}


%Für Test
%\usepackage{blindtext}


\addbibresource{references.bib}
\bibliography{references}
\makeglossaries
%
% acro.tex
%
% (c) 2024 Lukas Schöpf, OST Ostschweizer Fachhochschule
%
\newacronym{ost}{OST}{Ostschweizer Fachhochschule}
\newacronym{snr}{SNR}{Signal to Noise Ration}
\newacronym{cnn}{CNN}{Convolutional Neural Networks}
\newacronym{clip}{CLIP}{Contrastive Language-Image Pre-training}
\newacronym{align}{ALIGN}{A Large-scale ImagGe and Noisy-text embedding}
\newacronym{sam}{SAM}{Segment Anything}
\newacronym{hef}{HEF}{Hailo Executable File}
\newacronym{dfc}{DFC}{Dataflow Compiler}
\newacronym{pca}{PCA}{Principal Component Analysis}
\newacronym{har}{HAR}{Hailo Archive}
\newacronym{tse}{TSE}{Tiled Squeeze-and-Excite}
\newacronym{rnn}{RNN}{Recurrent neural network}
\newacronym{api}{API}{application programming interface}
\newacronym{gpu}{GPU}{Graphics processing unit}



% name des tex-file Reinschteiben welches man bearbeitet
%\includeonly{text/Entwicklung} % Auskommentiern um ganzes Dokument zu setzen

\begin{document}
    \pagenumbering{Roman}
     % öffnet titlepage.tex
    %
% titlepage.tex -- TeXfiles
%
% (c) 2024 Jakob Gierer & Lukas Schöpf, OST Ostschweizer Fachhochschule
%


\begin{titlepage}
    % Bigger symmetric margins
    \newgeometry{
        outer = 3cm, inner = 3cm, top=5cm,bottom=6cm
    }
    \centering
%    \vspace{4cm}

    % Titel
    {\huge \bfseries \sffamily Scene Understanding \par
     \normalfont\itshape On-Site Scene Understanding with Edge AI \par}
    \vspace{1cm}

    % Autoren
    {\large \textsl{Lukas Schöpf}}
    \par
    \vspace{1cm}

    % Informationen über Hochschule
    {\textsc Master Project \par}
    {Ostschweizer Fachhochschule \par}
    \today
    \vfill
    
    % Titelbild
    % Beispiel
%    \begin{figure}[h]
%        \centering
%        \includegraphics[width=3cm]{example-image-c}
%    \end{figure}

    % Oder Tikz
    % \resizebox{.9\linewidth}{!}{
        %   \input{figures/tikz/}
        % }
    \vfill

    % % Wichtige Informationen
    % \begin{tabular}{rl}
    %     \bfseries\sffamily Thema & Audioklassifizierung \\
    %     \bfseries\sffamily Fachgebiet          & Digitale Signalverarbeitung \\
    %     \bfseries\sffamily Betreuer       & Hannes Badertscher, Patrik Müller \\
        
    % \end{tabular}
    \restoregeometry    
\end{titlepage}


    %
% Abstrakt.tex
%
% (c) 2023 Jakob Gierer & Lukas Schöpf, OST Ostschweizer Fachhochschule
%

\chapter*{Abstrakt}
Test
	
	
    %
% Dankessagung.tex
%
% (c) 2023 Jakob Gierer & Lukas Schöpf, OST Ostschweizer Fachhochschule
%

\chapter*{Danksagung}

Wir möchten diese Gelegenheit nutzen, um unseren aufrichtigen Dank all jenen auszudrücken, die uns während der Erstellung unserer Bachelorarbeit unterstützt und inspiriert haben.
\vspace{11pt}

\noindent
Ein besonderer Dank gilt auch unseren Betreuungspersonen, Patrik Müller und Hannes Badertscher.
Ihre fachliche Expertise, ihre Geduld und ihre Ratschläge waren von unschätzbarem Wert für den Erfolg unserer Arbeit.
Wir sind dankbar für die konstruktive Kritik und die Unterstützung, die sie uns entgegengebracht haben.
\vspace{11pt}

\noindent 
Ein herzliches Dankeschön geht auch an Florian Baumgartner und Alain Keller, die uns bei der Anpassung und Benutzung ihres Codes unterstützten. 
\vspace{11pt}

\noindent
Ausserdem möchten wir unseren Familien für das Korrekturlesen unserer Bachelorarbeit danken. 
    \pagestyle{plain}
    \tableofcontents
    {
    	\linespread{0.7}\selectfont{}
    	\glsnogroupskiptrue
    	\printglossary[type=\acronymtype]
    }
    % \printglossary[type=\acronymtype]
    % add common macros
    %
% macros.tex
%
% (c) 2023 Jakob Gierer & Lukas Schöpf, OST Ostschweizer Fachhochschule
%

% Kopfzeilen
%
\renewcommand{\headrulewidth}{0.4pt}
\renewcommand{\chaptermark}[1]{%
    \markboth{#1}{}
}
\renewcommand{\sectionmark}[1]{
    \markright{#1}
}

\rhead{\leftmark}
\lhead{\rightmark}

%to set Numbers in Circles use \kreis{}
\newcommand{\kreis}[1]{\unitlength1ex
    \begin{picture}(2.5,2.5)
        \put(0.75,0.75){\circle{2.5}}\put(0.75,0.75){\makebox(0,0){#1}}
\end{picture}}

%for different colored Text
\definecolor{YellowOrange}{RGB}{255, 159, 43}
\definecolor{RoyalBlue}{RGB}{0, 122, 255}
\definecolor{ForestGreen}{RGB}{18, 159, 87}

% Beispiel umgebung erstellen
\newcounter{satz}[chapter]
\newenvironment{beispiel}{%
    \refstepcounter{satz}
    \begin{proof}[Beispiel \arabic{chapter}.\arabic{satz}]%
        \renewcommand{\qedsymbol}{$\bigcirc$}
    }{\end{proof}}
    \newpage
    \pagenumbering{arabic}
    \pagestyle{fancy}
    % ==================================================
    % Beispiel um externe Textdokumente hinzuzufügen 
    % \input{file}
    %%\input{text/Einleitung.tex}
    %%\input{text/Literaturrecherche.tex}
    % ==================================================
    % Quellen
    %%\input{text/Theorie.tex}
    %%\input{text/Systemübersicht.tex}
    %%\input{text/Mikrofonarray.tex}
    %%\input{text/Daten.tex}
    %%\input{text/Klassifizierung.tex}
    %%\input{text/Ergebnisse.tex}
    %%\input{text/Schlussfolgerung.tex}
    \newpage
    %%\input{common/selbstständigkeits.tex}
    \newpage
    % ==================================================
    % Verzeichnisse
    \pagenumbering{alph}
    \printbibliography
    \listoffigures
    \listoftables
    \newpage
    
\end{document}

