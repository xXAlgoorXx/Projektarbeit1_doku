\chapter{Results}
In this chapter the results are discussed and interpreted.
The conclusions taken are based of the other chapters in this work.

\section{Execute models on the edge}
To have a success full compilation of a network it is important to understand which part of network dont lose much accurcy due to quantisation.
In this project two implementations are tested.
One where as much as possibel is calculated on the edge.
The second tries to take in consideratio how good a given layer quantises.
Sadly both implementation lack of good results.
This means no statement in context of how good a layer quantises can be made.

All models suffered under a decrease of accuracy after quantisation.
This effect is sever with TinyCLIP
The suspision is that TinyCLIP has to few parameters to work propely after quantisation.
This would also explain why TinyCLIP isnt giving good results.
During weight inheritance some weights get masked.
This reduces the parameter count of a network by making it narrower.
In \cref{section:weightinheritance} it is mentiond that the authors of TinyCLIP found lot of redundancy in width when examing vision encoders and in the progress of make the model smaller reduced this redundancy.
This reduction seems lethal to the accuracy after quantisation.

The take away from this work is that models shouldn't be pruned or simplified before quantisation.
There is even the possibility to give this task to the \acrshort{dfc}.

\section{Working with Hailo}

Hailo has proven to have a good product with the Hailo 8L hardware accelerator.
The combination with the Raspberry Pi 5 makes it easy to develop in comparison to other solutions found.

The \acrshort{dfc} also contributes to simplicity of the solution by hailo.
The tutorials are built in the software packages.
Additional information can easly be found in the guide document of the \acrshort{dfc}.
In the guide are also code examples to use the hailo hardware.

Although nearly all examples from hailo are focused on image processing.
The help is handeld through the hailo community.
This is beneficial if other people have the same problem.
But on serius problems like the problem with the dimension swap (see \cref{implementation:sec:translation}) it fellt like no one of the hailo people which are answering in the forum understood enough to tell that the compilation of this specific network isnt possible at the moment.

The python \acrfull{api} can be found online or in the \acrshort{dfc} Guide.
Sadly it isnt discribed too good.
All code from hailo is open source.
There are many sources of examples to use the \acrshort{api} but these are again not described very well.
Most of the code has be understood by the programmer.
The other way would be asking in the forum of hailo.
There the time to answer can alter between 1-10 days.

The suspision arrises that full potential  of the \acrshort{dfc} the is not utilized because of the absence of a GPU.


\section{Conclusion}

To use qunatized models it isnt beneficial to reduce its parameter count before compilation.
Especialy with hailo because the \acrshort{dfc} is capable of pruning and optimization.
This is shown in the result as the CLIP models show the better results after quantisation than the TinyCLIP models.

It is beneficial to use apropoate hardware (\acrshort{gpu}) to speed up the compilation time.
There are also cetrain functions of the \acrshort{dfc} which arent possible to make without a \acrshort{gpu} like optimization of a network.

To really use the full potential of the Hailo hardware accelerator and CLIP one should wait until the implementation to use transformers is complete. 

\section{Outlook}

The next steps should be to wait for the next update when hailo implements the usage of Transformers.
It is important to keep in mind that a suspision from this project is that matrix multiplications dont quantise well.
The self attention layer from Transformers are basicly matrix multiplications (see \cref{equ:selfattention})
To further incerase the scene understanding capabilities SAM\cite{sam} could be used as preprocessing.
\acrshort{sam} returns segments of pictures.
This in compination with CLIP could be used to incerase the accuracy of classification.
\acrshort{sam} could for example isolate things in a office for example a printer or a computer which then gets classified by CLIP.
To use this most likely 2 hardware accelerators, one for each model.
This means that the Raspberry Pi 5 wouldnt be capable of this task because it has only one M.2 slot.

